% This is samplepaper.tex, a sample chapter demonstrating the
% LLNCS macro package for Springer Computer Science proceedings;
% Version 2.20 of 2017/10/04
%
\documentclass[runningheads]{llncs}
%
\usepackage{graphicx}
% Used for displaying a sample figure. If possible, figure files should
% be included in EPS format.
%
% If you use the hyperref package, please uncomment the following line
% to display URLs in blue roman font according to Springer's eBook style:
% \renewcommand\UrlFont{\color{blue}\rmfamily}

\begin{document}
%
\title{Predicting Customer Behavior in Consumer Durable Market}
%
%\titlerunning{Abbreviated paper title}
% If the paper title is too long for the running head, you can set
% an abbreviated paper title here
%
\author{Sylvester Ranjith F \inst{1} 
\and V.Vijayarajan\inst{2}}
%
% First names are abbreviated in the running head.
% If there are more than two authors, 'et al.' is used.
%
\institute{VIT University, Vellore, Tamil Nadu 632014}
%
\maketitle              % typeset the header of the contribution
%
\begin{abstract}

The idea behind this project is to analyse the data of a consumer durable store and predict the behaviour of customers based on parameters like frequency of visits,purchasing habits  and factors like the location of the branches in relation to their competitor stores.The objective of this project is to predict the reason for a customer not returning to the store and suggesting methods to the store by which the store should influence the customers by enticing them with offers or loyalty bonuses.Customer Behaviour can be influenced by varied factors like pricing,more offers provided by the competitor exhibiting customer disloyalty.The dataset taken into consideration is
data from a real retail consumer durable store,”Sathya Agencies,Pvt.Ltd”, A retail
giant popular in Tamil Nadu,India.The idea of this project is to turn the non-returning customers into frequent visitors so as to increase brand loyalty and value.


\keywords{Customer Behaviour  \and Human Behaviour \and Prediction  \and  Market-Patterns \and  Machine-Learning  \and  Supervised Learning}
\end{abstract}
%
%
%
\section{Introduction}

\paragraph{Sample Heading (Fourth Level)}
The contribution should contain no more than four levels of
headings. Table~\ref{tab1} gives a summary of all heading levels.

\section{Review of Literature}

\paragraph{Sample Heading (Fourth Level)}
The contribution should contain no more than four levels of
headings. Table~\ref{tab1} gives a summary of all heading levels.



Date of Submission - 30 Jan 2019





%
% ---- Bibliography ----
%
% BibTeX users should specify bibliography style 'splncs04'.
% References will then be sorted and formatted in the correct style.
%
% \bibliographystyle{splncs04}
% \bibliography{mybibliography}
%
\begin{thebibliography}{8}
\bibitem{ref_article1}
Author, F.: Article title. Journal \textbf{2}(5), 99--110 (2016)

\bibitem{ref_lncs1}
Author, F., Author, S.: Title of a proceedings paper. In: Editor,
F., Editor, S. (eds.) CONFERENCE 2016, LNCS, vol. 9999, pp. 1--13.
Springer, Heidelberg (2016). \doi{10.10007/1234567890}

\bibitem{ref_book1}
Author, F., Author, S., Author, T.: Book title. 2nd edn. Publisher,
Location (1999)

\bibitem{ref_proc1}
Author, A.-B.: Contribution title. In: 9th International Proceedings
on Proceedings, pp. 1--2. Publisher, Location (2010)

\bibitem{ref_url1}
LNCS Homepage, \url{http://www.springer.com/lncs}. Last accessed 4
Oct 2017
\end{thebibliography}
\end{document}
